\documentclass[t]{beamer}
\usepackage[utf8]{inputenc}  % to be able to type unicode text directly
\usepackage{inconsolata}     % for a nicer (e.g. non-courier) tt family font
\usepackage{array}           % to fine-tune tabular spacing
\usepackage{bbm}             % for blackboard 1
\usepackage{graphicx}        % to include images
\usepackage{graphbox}        % to fine-tune image alignment
\usepackage{animate}         % to include animated images
\usepackage{soul}            % for colored strikethrough
%\usepackage{bbding}          % for Checkmark and XSolidBrush
\usepackage{hyperref,url}

\colorlet{darkgreen}{black!50!green}  % used for page numbers
\colorlet{darkred}{black!50!red}  % used for page numbers
\colorlet{darkblue}{black!30!blue}  % used for page numbers
\colorlet{lightblue}{white!50!blue}  % used for page numbers
\definecolor{term}{rgb}{.9,.9,.9}     % used for code insets

\setlength{\parindent}{0em}  % no paragraph indentation
\setlength{\parskip}{1em}    % paragraph spacing


% coco's macros
\def\ds{\displaystyle}
\def\R{\textbf{R}}
\def\F{\mathcal{F}}
\def\x{\textbf{x}}
\def\y{\textbf{y}}
\def\u{\mathbf{u}}
\def\Z{\textbf{Z}}
\def\d{\mathrm{d}}
\def\ud{\mathrm{d}}
\newcommand{\abs}[1]{\left| #1 \right|}
\newcommand{\Abs}[1]{\left\| #1 \right\|}
\DeclareMathOperator*{\argmin}{arg\,min}
\DeclareMathOperator*{\argmax}{arg\,max}
\newcommand{\reference}[1] {{\scriptsize \color{gray}  #1 }}
\newcommand{\referencep}[1] {{\tiny \color{gray}  #1 }}
\newcommand{\unit}[1] {{\tiny \color{gray}  #1 }}

% disable spacing around verbatim
\usepackage{etoolbox}
\makeatletter\preto{\@verbatim}{\topsep=0pt \partopsep=0pt }\makeatother

% disable headings, set slide numbers in green
\mode<all>\setbeamertemplate{navigation symbols}{}
\defbeamertemplate*{footline}{pagecount}{\leavevmode\hfill\color{darkgreen}
   \insertframenumber{} / \inserttotalframenumber\hspace*{0ex}\vskip0pt}

%% select red color for strikethrough
\makeatletter
\newcommand\SoulColor{%
  \let\set@color\beamerorig@set@color
  \let\reset@color\beamerorig@reset@color}
\makeatother
\newcommand<>{\St}[1]{\only#2{\SoulColor\st{#1}}}
\setstcolor{red}

% make everything monospace
\renewcommand*\familydefault{\ttdefault}





\begin{document}


\addtocounter{framenumber}{-1}
\begin{frame}[plain,fragile]
\Large
\begin{verbatim}






  the Riesz semigroup in image processing






mnhrdt
gtti 6/4/2023
\end{verbatim}
\end{frame}



% intro 1 newton vs. others
\begin{frame}
INTRO\\
=====

{\bf Question:}
Who is the best scientist of all time?\\
\pause
{\bf Answer:}
Easy! It's Isaac Newton\\
\begin{center}
\includegraphics[height=0.3\textheight]{f/newton.png}
\end{center}

\pause
%\vfill

{\bf Question:}
Who is the {\color{blue} second-best} scientist of all time?\\
\pause
{\bf Answer:}
Hmmm...\\
\color{blue}
\small
%Archimedes, von Neumann, Einstein, Darwin, Pasteur,
%Curie, Galileo, Lavoisier, Maxwell, $\ldots$
\includegraphics[width=\textwidth]{f/scientists1c.png}
\includegraphics[width=\textwidth]{f/scientists2.png}

\end{frame}

% intro 2 gaussian kernel vs. riesz kernel
\begin{frame}
INTRO\\
=====

{\bf Question:}
%\\
What is the best convolution kernel?\\
\pause
{\bf Answer:}
	Easy! {\color{darkblue}The Gaussian kernels}
%\[
%	\color{darkblue}
%	G_\sigma(x,y)\ =\  {\color{lightblue}\frac1{2\pi\sigma^2}} \exp\frac{-(x^2+y^2)}{2\sigma^2}
%	\pause
%	\ =\  {\color{lightblue}\frac1{\sigma^d\sqrt{2\pi}^d}}
%	\exp\frac{-\Abs{\vec x}^2}{2\sigma^2}
%\]
\begin{tabular}{ll}
	\fbox{
		\(\displaystyle\color{darkblue}
	G_\sigma(x,y)\ =\  {\color{lightblue}\frac1{2\pi\sigma^2}} \exp\frac{-(x^2+y^2)}{2\sigma^2}
		\)
	}&
	\includegraphics[align=c,width=0.29\textwidth]{f/shot_gauss.png}
\end{tabular}

\pause
%\vfill

{\bf Question:}
What is the second-best convolution kernel?\\
\pause
{\bf Answer:}
	Easy! {\color{darkred}The Riesz potentials}
%\[
%	\color{darkred}
%	R_\alpha(x,y)\ =\  {\color{pink}k_\alpha}
%	\left(x^2+y^2\right)^{\frac{\alpha-2}2}
%	\pause
%	\ =\  {\color{pink}k_\alpha}
%	\frac1{\Abs{\vec x}^{d-\alpha}}
%\]
\begin{tabular}{ll}
	\fbox{
		\(\displaystyle\color{darkred}
	R_\alpha(x,y)\ =\  {\color{pink}k_\alpha}
	\left(x^2+y^2\right)^{\frac{\alpha-2}2}
		\)
	}&
	\includegraphics[align=c,width=0.29\textwidth]{f/shot_land.png}
\end{tabular}
\pause

%SCRIPT blur g 5 f/x.png o/x_g5.png
%SCRIPT blur g 15 f/x.png o/x_g15.png
%SCRIPT blur g 60 f/x.png o/x_g60.png
%SCRIPT fft 1 f/x.png|plambda ':R 1 ^ /'|ifft|qauto -i -p 0.1 - o/x_r1.png
%SCRIPT fft 1 f/x.png|plambda ':R 2 ^ /'|ifft|qauto -i -p 0.1 - o/x_r2.png
%SCRIPT fft 1 f/x.png|plambda ':R 3 ^ /'|ifft|qauto -i -p 0.1 - o/x_r3.png
%SCRIPT fft 1 f/x.png|plambda ':R -0.5 ^ /'|ifft|qauto -i -p 5 - o/x_rm05.png
%SCRIPT fft 1 f/x.png|plambda ':R -1 ^ /'|ifft|qauto -i -p 5 - o/x_rm1.png
%SCRIPT fft 1 f/x.png|plambda ':R -2 ^ /'|ifft|qauto -i -p 5 - o/x_rm2.png

\vspace{-1.5em}
\begin{center}
	\setlength{\tabcolsep}{1pt}
	\renewcommand{\arraystretch}{0.5}
	\begin{tabular}{ccccccc}
		\scriptsize $u$ &
		\scriptsize ${\color{darkblue}G_5} * u$ &
		\scriptsize ${\color{darkblue}G_{15}} * u$ &
		\scriptsize ${\color{darkblue}G_{60}} * u$ &
		\scriptsize ${\color{darkred}R_1} * u$ &
		\scriptsize ${\color{darkred}R_2} * u$ &
		\scriptsize ${\color{darkred}R_3} * u$ \\
		\includegraphics[width=0.14\textwidth]{f/x.png}&
		\includegraphics[width=0.14\textwidth]{o/x_g5.png}&
		\includegraphics[width=0.14\textwidth]{o/x_g15.png}&
		\includegraphics[width=0.14\textwidth]{o/x_g60.png}&
		\includegraphics[width=0.14\textwidth]{o/x_r1.png}&
		%\includegraphics[width=0.14\textwidth]{o/x_rm05.png}&
		%\includegraphics[width=0.14\textwidth]{o/x_rm1.png}\\
		\includegraphics[width=0.14\textwidth]{o/x_r2.png}&
		\includegraphics[width=0.14\textwidth]{o/x_r3.png}\\
	\end{tabular}
\end{center}

\end{frame}


% why is the gaussian kernel great (laplace quote)
\begin{frame}
WHY IS THE GAUSSIAN KERNEL SO NATURAL\\
=====================================

	{\bf Statistics:} The Central Limit theorem\\
	%Iterating positive linear filters has the same effect as convolving with a Gaussian
	{\color{darkblue} A deep CNN with positive weights does a Gaussian blur.}


	\pause
	{\bf Analysis:} The heat equation {\small
	$\begin{cases}\frac{\partial}{\partial t}u
	=\Delta u&\\u(x,0)\!=\!f(x)\end{cases}\!\!\!\!\!\!\implies\!\!u(x,t)\!=\!{\color{darkblue}G_t}*f$}

	\pause
	{\bf Geometry:} (Herschel 1850, Maxwell 1860)\\
	The only densities that are radial and separable
	are~$\color{darkblue}G_\sigma$:\\
	If
	\(\small\begin{cases}
		G(x,y)=G(\sqrt{x^2+y^2},0)\\
		G(x,y)=g(x)g(y)
	\end{cases}\)
	\hspace{-1em}then
	%$g(x)g(y)=g\left(\sqrt{x^2+y^2}\right)g(0)$ and thus
	$\exists\alpha,\beta>0$:
	$\color{darkblue}g(x)=\alpha e^{-\beta x^2}$.

	\begin{tabular}{llll}
		\includegraphics[height=0.26\textheight]{f/john_herschel2.jpg}&
		\includegraphics[height=0.26\textheight]{f/john_herschel3.jpg}&
		\includegraphics[height=0.26\textheight]{f/john_herschel1.jpg}&
		\includegraphics[height=0.26\textheight]{f/herschel_darwin.jpg}
	\end{tabular}

\end{frame}


% teaser snapshot (fake clouds)
\begin{frame}
SOME LIMITATIONS OF THE GAUSSIAN KERNEL\\
=======================================

	{\bf Non reversible:} You cannot recover~$f$
	from~${\color{darkblue}G_\sigma} *f$

	\pause
	{\color{darkred}
	For Riesz kernels, however: $R_\alpha*R_\beta = R_{\alpha+\beta}\ \ $ \fbox{\color{darkred}$\forall\alpha,\beta\in\textbf{R}$}
	}

	\pause
	{\bf It just blurs at a single scale:}\\
	\scriptsize\vspace{0.3em} \begin{tabular}{cc}
		\includegraphics[height=0.48\textwidth]{o/cloud_gauss.png}&
		\includegraphics[height=0.48\textwidth]{o/cloud_riesz.png}\\
		Gaussian clouds ($\color{darkblue}\sigma=60$) &
		Riesz clouds ($\color{darkred}\alpha=1.6$) \\
	\end{tabular}
\end{frame}

%SCRIPT plambda f/calanques.jpg randu|fft|plambda ":R 1.6 ^ /"|ifft|plambda - "x x%q90 - x%a x%i - / 45 * atan pi / 2 * 255 *" -o o/cloud_riesz.png
%SCRIPT plambda f/calanques.jpg randu|blur g 40|plambda - "x x%q90 - x%a x%i - / 45 * atan pi / 2 * 255 *" -o o/cloud_gauss.png


% plan
% 1. definition and properties of the Riesz semigroup
% 2. comparison with the Gaussian semigroup
% 3. applications
% 3.1. retinex
% 3.2. differentiable poisson editing
% 3.3. shepard interpolation
% 3.4. cloud simulation
\begin{frame}
OUTLINE\\
=======

	1. Definition and properties of~$\color{darkred}R_\alpha$ {\color{gray}(Riesz semigroup)}

	2. Comparison with~$\color{darkblue}G_\sigma$ {\color{gray}(Gaussian semigroup)}

3. Applications in image processing\\
\color{gray}
$\quad$ 3.1. Really multi-scale Retinex\\
$\quad$ 3.2. Differentiable Poisson editing\\
$\quad$ 3.3. Inverse Distance Weighted interpolation\\
$\quad$ 3.4. Cloud simulation

\end{frame}


% definition and visualization, side by side
\begin{frame}
DEFINITION OF THE RIESZ KERNEL\\
==============================

%	{\bf Definition:}
	\fbox{
	\(\displaystyle
\color{darkred}
R_\alpha(x,y)\ =\  {\color{pink}k_\alpha}
	\left(x^2+y^2\right)^{\frac{\alpha-2}2}
	\pause
	\ =\  {\color{pink}k_\alpha}
	\frac1{\Abs{\vec x}^{d-\alpha}}
	\)}
$\color{darkred}\ \ \forall\alpha\in\textbf{R}$

%{\small
%	For~$\alpha\in\mathbf{R}$, $\color{darkred}R_\alpha$ acts over
%	\ul{images of zero mean} by convolution
%}

\begin{center}
	\setlength{\tabcolsep}{1pt}
	\renewcommand{\arraystretch}{0.5}
	\begin{tabular}{ccc}
		\includegraphics[width=0.24\textwidth]{o/x_rm2.png}&
		\includegraphics[width=0.24\textwidth]{o/x_rm1.png}&
		\includegraphics[width=0.24\textwidth]{o/x_rm05.png}\\
		\scriptsize ${\color{darkred}R_{-2}} * u=-\Delta u$ &
		\scriptsize ${\color{darkred}R_{-1}} * u$ &
		\scriptsize ${\color{darkred}R_{-0.5}} * u$ \\
		&&\\
		\includegraphics[width=0.24\textwidth]{f/x.png}&
		\includegraphics[width=0.24\textwidth]{o/x_r1.png}&
		\includegraphics[width=0.24\textwidth]{o/x_r2.png}\\
		\scriptsize ${\color{darkred}R_0} * u = u$ &
		\scriptsize ${\color{darkred}R_1} * u$ &
		\scriptsize ${\color{darkred}R_2} * u = -\Delta^{-1} u$ \\
	\end{tabular}
\end{center}

\end{frame}


% fractional derivatives (as from wikipedia)
% [note: riesz kernel are fractional derivatives in dimension 2]
% https://en.wikipedia.org/wiki/Riemann%E2%80%93Liouville_integral
\begin{frame}
FRACTIONAL DERIVATIVES\\
======================

	{\bf Definition: }(Riemann-Liouville integral)
	\[
		I^\alpha f(x)=\frac1{\Gamma(\alpha)}\int^xf(t)(x-t)^{\alpha -1}\ud t
		\]
	{\bf Properties:} {\color{gray}(elementary)}
	\[
		I^1 f = \int f
		\qquad
		\qquad
		\frac{\ud}{\ud x}I^{\alpha+1} f(x)=I^\alpha f
		\qquad
		\qquad
		I^\alpha I^\beta = I^{\alpha+\beta}
		\]

		Extending~$I^\alpha$ for~$\alpha\in\textbf{R}$ we get a continuous interpolation between derivatives and integrals.

\animategraphics[autoplay,loop,width=0.3\linewidth]{10}{f/a/fracdev-}{0}{75}

\end{frame}


% recall properties of fourier transform
\begin{frame}
PROPERTIES OF THE FOURIER TRANSFORM\\
===================================

For~$f:\R^d\to\R$ define~\fbox{$\ds\widehat
	f(\y)=\int_{\R^d} f(\x)e^{-i\x\cdot\y}\ud \x$}

{\bf Properties:}
	(1) $\ds\widehat{\frac{\partial }{\partial x_k}f} (\y)= iy_k\widehat{f}(\y)$

	(2) If~$f$ is~$\sigma$-homogeneous then~$\widehat f$ is~$(-d-\sigma)$-homogeneous.\\
	{\color{gray}\small
	$f(\lambda\x)=\lambda^\sigma f(\x)\ \forall\lambda>0\ \implies\ %
	\widehat{f}(\lambda\y)=\lambda^{-d-\sigma}\widehat{f}(\y)\ \forall\lambda>0$
	}

	(3) If~$f$ is radial then~$\widehat f$ is radial.

	\pause
	{\bf Corollary:}
	$\widehat{\color{darkred}R_\alpha} \propto \color{darkred}R_{d-\alpha}$\\
	{\small\color{gray}{\bf Proof:}~$R_\alpha(\x)=\Abs{\x}^{\alpha-d}$ is
	radial and~$(\alpha-d)$-homogeneous.}

	\pause
	{\bf Caveats:}
	Non-integrability, Diracs at the origin, ...

\end{frame}

%
\begin{frame}
TECHNICAL DETAILS (\only<1>{1}\only<2>{2}/2)\\
=======================

\only<1>{
\begin{tabular}{ll}
	\includegraphics[height=0.8\textheight]{f/kellog.jpg}&
	\includegraphics[height=0.8\textheight]{f/hormander.png}\\
\end{tabular}}\only<2>{
\includegraphics[width=0.6\textwidth]{f/jaffe1.png}\\
	\vspace{-1em}
. . .

\includegraphics[width=0.95\textwidth]{f/jaffe2.png}
}


\end{frame}


% definition in the spectral domain (by means of symmetries)
\begin{frame}
SPECTRUM OF THE RIESZ POTENTIAL\\
===============================

	{\bf Idea:}
	Generalize~$\widehat{f^{(n)}}(y)=(iy)^n\widehat f(y)$ as
	$\widehat{f^{(\alpha)}}(y)=\abs{y}^\alpha\widehat f(y)$.

	{\bf New definition:}
	$\color{darkred}R_\alpha$ is the ``Fourier multiplier''
	%with spectrum
	\begin{center}
	\fbox{$\ds\color{darkred}\widehat{R_\alpha}(\y)=\frac1{\Abs{\y}^\alpha}$}
	\end{center}

	{\bf Formal properties: } (now elementary)\\
	$$\color{darkred}R_\alpha* R_\beta = R_{\alpha+\beta}$$
	$$\color{darkred}R_{-2} = \Delta
	\qquad
	\qquad
	\color{darkred}R_0 = \delta
	\qquad
	\qquad
	\color{darkred}R_{2} = \Delta^{-1}$$


\end{frame}

% fixed points
\begin{frame}
CURIOSITY: FIXED POINTS OF THE FOURIER TRANSFORM\\
================================================

{\bf d=1:} $\ds \frac1{\sqrt{\abs{x}}}$

{\bf d=2:} $\ds\color{darkred}\frac1{\sqrt{x^2+y^2}}\qquad\qquad$
{\color{gray} (Land kernel)}

{\bf d=3:} $\ds \frac1{\sqrt{x^2+y^2+z^2}^3}\qquad$
{\color{gray} (Newton-like potential)}

{\bf d=4:} $\ds \frac1{x^2+y^2+z^2+t^2}\qquad$
{\color{gray} (antilaplacian)}

\vfill

Proof:
$\widehat{\Abs{\x}^s} \propto  \Abs{\y}^{-d-s}
\qquad
\qquad
s=-d-s\iff s=-d/2
$


\end{frame}

% visual examples
\begin{frame}
THIS SHOULD BE AN ANIMATION\\
===========================

\begin{center}
	\setlength{\tabcolsep}{1pt}
	\renewcommand{\arraystretch}{0.5}
	\begin{tabular}{ccc}
		\includegraphics[width=0.31\textwidth]{o/x_rm2.png}&
		\includegraphics[width=0.31\textwidth]{o/x_rm1.png}&
		\includegraphics[width=0.31\textwidth]{o/x_rm05.png}\\
		\scriptsize ${\color{darkred}R_{-2}} * u=-\Delta u$ &
		\scriptsize ${\color{darkred}R_{-1}} * u$ &
		\scriptsize ${\color{darkred}R_{-0.5}} * u$ \\
		&&\\
		\includegraphics[width=0.31\textwidth]{f/x.png}&
		\includegraphics[width=0.31\textwidth]{o/x_r1.png}&
		\includegraphics[width=0.31\textwidth]{o/x_r2.png}\\
		\scriptsize ${\color{darkred}R_0} * u = u$ &
		\scriptsize ${\color{darkred}R_1} * u$ &
		\scriptsize ${\color{darkred}R_2} * u = -\Delta^{-1} u$ \\
	\end{tabular}
\end{center}
\end{frame}

% commutation with zoom
\begin{frame}
COMMUTATION WITH ZOOM\\
=====================

Let~$H_\lambda f(\x) := f(\lambda\x).\quad$
{\color{gray}(zoom of factor $\lambda$)}

We have:
\begin{center}
	\fbox{
		\(\color{darkred}
			R_\alpha H_\lambda\ \propto\ H_\lambda R_\alpha
		\)
	}
\end{center}

Contrast it with:
\begin{center}
	\fbox{
		\(\color{darkblue}
			G_\sigma H_\lambda\ \propto\ H_\lambda G_{\lambda\sigma}
		\)
	}
\end{center}

The {\color{darkblue}Gaussian semigroup} is co-variant with scale.

The {\color{darkblue}Riesz semigroup} is \ul{invariant} with scale.

(It's NOT a  scale-space, each level contains information about all the scales.)

\end{frame}

% continuous interpolation between laplacian, retinex, image, blurry image
\begin{frame}
INTERESTING POINTS IN THE RIESZ SEMIGROUP FOR {\bf d=2}\\
=====================================================

$\ds\color{darkred}\widehat{R_\alpha}(\y)=\frac1{\Abs{\y}^\alpha}$
is defined for all~$\alpha\in\R$

\vfill

$\ds\color{darkred}\widehat{R_{-2}}(\y)=\Abs{\y}^2$ is the Laplacian:
$R_{-2}u=
\frac{\partial^2}{\partial x^2}u
+\frac{\partial^2}{\partial y^2}u
$

$\ds\color{darkred}\widehat{R_{-1}}(\y)=\Abs{\y}$
a sort of ``symmetrized'' first-derivative

$\ds\color{darkred}\widehat{R_0}(\y)=1$
the identity

\fbox{
$\ds\color{darkred}\widehat{R_1}(\y)=\frac1{\Abs{\y}}$
Land's kernel~$\ds\color{darkred} R_1(\x)=\frac1{\Abs{\x}}$
}

$\ds\color{darkred}\widehat{R_2}(\y)=\Delta^{-1}$ the anti-laplacian



\end{frame}


% first derivatives
%\begin{frame}
%GET FIRST DERIVATIVES FROM THE LAPLACIAN\\
%========================================
%
%\end{frame}


% closely related construction: riesz potentials, riesz transform
% https://en.wikipedia.org/wiki/Riesz_potential
% https://en.wikipedia.org/wiki/Riesz_transform
%\begin{frame}
%SIMILAR CONSTRUCTIoNS\\
%=====================
%
%\end{frame}
%
%\begin{frame}
%FORMAL DEFINITION, CONSTANTS\\
%============================
%
%\end{frame}


% the Land kernel (particular case for σ=1)
\begin{frame}
THE LAND KERNEL $\color{darkred}k=R_1$\\
======================

%\small
\begin{tabular}{l|l}
	{\color{darkblue}Gauss} & {\color{darkred}Land} \\
	\hline
	&\\
	{\color{white} v} $\color{darkblue}G_\sigma(x,y)=\frac{1}{2\pi\sigma^2}\exp\frac{x^2+y^2}{-2\sigma^2}$ &
	{\color{white} v} $\color{darkred}k(x,y)=\frac{1}{\sqrt{x^2+y^2}}$ \\
	&\\
	{\color{darkgreen} v} $C^\infty$ function &
	{\color{red} x} vertical asymptote in 0 \\
	{\color{darkgreen} v} radial &
	{\color{darkgreen} v} radial \\
	{\color{darkgreen} v} séparable &
	{\color{red} x} non-séparable \\
	{\color{darkgreen} v} probability density~$1$ &
	{\color{red} x} locally intégrable \\
	{\color{red} x} one parameter $\sigma$ &
	{\color{darkgreen} v} no parameters \\
	{\color{darkgreen} v} $G_{\sigma}*G_{\mu}=g_{\tiny\sqrt{\sigma^2+\mu^2}}$ &
	{\color{darkgreen} v} $k*k=\Delta^{-1}$ \\
	{\color{darkgreen} v} $\widehat{G_\sigma}\ \propto\  G_{\sigma^{-1}}$ &
	{\color{darkgreen} v} $\widehat{k} = k$ \\
	%{\color{darkgreen} v} $\Delta g_\sigma = \textrm{chapeau mexicain}$ &
	%{\color{darkgreen} v} $\Delta k=k^3-\delta\quad \textrm{(retinex)} $ \\
	{\color{red} x} $G_\sigma*$ non-inversible &
	{\color{darkgreen} v} $k*$ inversible \\
	%{\color{darkgreen} v} $g_\sigma*(Z_\lambda u)=Z_\lambda(g_{\sigma\lambda}*u)$ &
	%{\color{darkgreen} v} $k*(Z_\lambda u) = Z_\lambda(\lambda k* u)$\\
	{\color{darkgreen} v} \it scale-covariant&
	{\color{darkgreen} v} \it scale-invariant\\
	{\color{darkgreen} v} $G_\sigma*u$ scale space &
	{\color{darkgreen} v} $k*u$ contains all scales \\
%	{\color{darkgreen} v} temps réel (300 FPS) &
%	{\color{red} x} assez rapide (15 FPS) \\
\end{tabular}
\end{frame}


% implementation: imscript (with fft), imscript (with blur), python
\begin{frame}[fragile]
IMPLEMENTATION\\
==============

{\bf imscript}:\\
{\color{darkred}\small
\verb+blur riesz 1.6 barbara.png barbara_R16.png+
}

\vfill

{\bf plambda}:\\
{\color{darkred}\small
\verb+cat barbara.png | fft | plambda ':R 1.6 ^ /' | ifft > barbara_R16.npy+
}

\vfill

{\bf python}:

\includegraphics[width=0.8\linewidth]{f/pyshot.png}

\end{frame}


%% comparison table gauss/riesz/land
%\begin{frame}
%COMPARISON TABLE\\
%================
%
%\end{frame}




% 3. applications
\begin{frame}
APPLICATIONS\\
============

3. Applications\\
$\quad$ 3.1. Retinex\\
$\quad$ 3.2. Differentiable Poisson editing\\
$\quad$ 3.3. Shepard interpolation\\
$\quad$ 3.4. Cloud simulation\\

\vfill

\begin{tabular}{ccc}
	\includegraphics[width=0.31\linewidth]{f/calanques.jpg}&
	\includegraphics[width=0.31\linewidth]{f/cloud.png}&
	\includegraphics[width=0.31\linewidth]{f/composite.jpg}\\
	original image & riesz clouds & composite\\
\end{tabular}

\end{frame}


% 3.1. multi-scale retinex (three-gaussians shit, are an approximation of land)
\begin{frame}
MULTI-SCALE RETINEX\\
===================

\includegraphics[width=0.61\linewidth]{f/msreti.png}

{\bf Method:}~$u\mapsto SCB(u - G_6*u - G_{36}*u - G_{200}*u)$

\pause
{\bf Interpretation:}
This is just a bad approximation of a Riesz kernel~$R_{-0.5}$\\
	\begin{tabular}{ccc}
		\includegraphics[width=0.25\textwidth]{f/x.png}&
		\includegraphics[width=0.25\textwidth]{o/x_rm05.png}&
		\includegraphics[width=0.25\textwidth]{o/x_rm1.png}\\
		\scriptsize $u$ &
		\scriptsize ${\color{darkred}R_{-0.5}} * u$ &
		\scriptsize ${\color{darkred}R_{-1}} * u$ \\
	\end{tabular}

\end{frame}


% 3.2. - antilaplacian = land * land
\begin{frame}
DIFFERENTIABLE POISSON EDITING
==============================

Poisson editing is a standard technique:\\
\includegraphics[width=0.56\linewidth]{f/poisbear.png}



Since~$\Delta^{-1}=k*k=\color{darkred}R_{-2}$, it can be solved by local convolutions using riesz filters
\end{frame}


% 3.3. shepard interpolation
% https://en.wikipedia.org/wiki/Inverse_distance_weighting
\begin{frame}
SHEPARD INTERPOLATION (IDW)\\
===========================

Inverse-distance weighted interpolation is a standard technique.

\includegraphics[width=\linewidth]{f/shepard.png}

It is just Riesz (Land) filtering of the function
\[
	u(\x)=\sum_{i=1}^n a_i \delta(\x-{\x}_i)
\]

\end{frame}


% 3.4. sobolev fractals, cloud simulation, other textures
\begin{frame}
SOBOLEV FRACTALS, CLOUD SIMULATION\\
==================================

\only<1>{\includegraphics[width=\linewidth]{f/frac1.png}}
\only<2>{\includegraphics[width=\linewidth]{f/frac2.png}}
\small
$$\color{violet}
f(x)=\sum_n \frac{e^{i\theta_n}}{|n|^{-p}} e^{inx},
\qquad
\theta_n\in\mathcal{U}(0,2\pi)
$$


\end{frame}


% gaussian mandelbrot quote
%\begin{frame}
%CLT AND ITS HEURISTIC CONVERSE\\
%==============================
%
%\end{frame}


% conclusion: links to slides, imscript, python function
\begin{frame}
CONCLUSION\\
==========

{\bf Open problems:}

1. Find a cool PDE satisfied by the Riesz semigroup.

2. Fast implementation as a linear u-net (ongoing work).

3. Understand why don't everybody uses this.

\vfill

{\bf Motto:}

Images are multi-scale, we must use multi-scale filters!

\end{frame}


% colophon: makefile that runs experiments and builds everything
\begin{frame}
COLOPHON\\
========

\end{frame}


\end{document}


% vim:sw=2 ts=2 spell spelllang=en:
