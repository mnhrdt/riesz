\documentclass[t]{beamer}
\usepackage[utf8]{inputenc}  % to be able to type unicode text directly
\usepackage{inconsolata}     % for a nicer (e.g. non-courier) tt family font
\usepackage{array}           % to fine-tune tabular spacing
\usepackage{bbm}             % for blackboard 1
\usepackage{graphicx}        % to include images
%\usepackage{animate}         % to include animated images
\usepackage{soul}            % for colored strikethrough
%\usepackage{bbding}          % for Checkmark and XSolidBrush
\usepackage{hyperref,url}

\colorlet{darkgreen}{black!50!green}  % used for page numbers
\definecolor{term}{rgb}{.9,.9,.9}     % used for code insets

\setlength{\parindent}{0em}  % no paragraph indentation
\setlength{\parskip}{1em}    % paragraph spacing


% coco's macros
\def\R{\mathbf{R}}
\def\F{\mathcal{F}}
\def\x{\mathbf{x}}
\def\y{\mathbf{y}}
\def\u{\mathbf{u}}
\def\Z{\textbf{Z}}
\def\d{\mathrm{d}}
\DeclareMathOperator*{\argmin}{arg\,min}
\DeclareMathOperator*{\argmax}{arg\,max}
\newcommand{\reference}[1] {{\scriptsize \color{gray}  #1 }}
\newcommand{\referencep}[1] {{\tiny \color{gray}  #1 }}
\newcommand{\unit}[1] {{\tiny \color{gray}  #1 }}

% disable spacing around verbatim
\usepackage{etoolbox}
\makeatletter\preto{\@verbatim}{\topsep=0pt \partopsep=0pt }\makeatother

% disable headings, set slide numbers in green
\mode<all>\setbeamertemplate{navigation symbols}{}
\defbeamertemplate*{footline}{pagecount}{\leavevmode\hfill\color{darkgreen}
   \insertframenumber{} / \inserttotalframenumber\hspace*{2ex}\vskip0pt}

%% select red color for strikethrough
\makeatletter
\newcommand\SoulColor{%
  \let\set@color\beamerorig@set@color
  \let\reset@color\beamerorig@reset@color}
\makeatother
\newcommand<>{\St}[1]{\only#2{\SoulColor\st{#1}}}
\setstcolor{red}

% make everything monospace
\renewcommand*\familydefault{\ttdefault}





\begin{document}


\addtocounter{framenumber}{-1}
\begin{frame}[plain,fragile]
\Large
\begin{verbatim}






  the Riesz semigroup in image processing






mnhrdt
gtti 6/4/2023
\end{verbatim}
\end{frame}



% intro 1 newton vs. others
\begin{frame}
INTRO\\
=====

\end{frame}

% intro 2 gaussian kernel vs. riesz kernel
\begin{frame}
INTRO\\
=====

\end{frame}


% why is the gaussian kernel great (laplace quote)
\begin{frame}
WHY IS THE GAUSSIAN KERNEL SO NATURAL\\
=====================================

\end{frame}


% teaser snapshot (fake clouds)
\begin{frame}
SOME LIMITATIONS OF THE GAUSSIAN KERNEL\\
=======================================

\end{frame}


% plan
% 1. definition and properties of the Riesz semigroup
% 2. comparison with the Gaussian semigroup
% 3. applications
% 3.1. retinex
% 3.2. differentiable poisson editing
% 3.3. shepard interpolation
% 3.4. cloud simulation
\begin{frame}
OUTLINE\\
=======

\end{frame}


% definition and visualization, side by side
\begin{frame}
DEFINITION OF THE RIESZ KERNEL\\
==============================

\end{frame}


% fractional derivatives (as from wikipedia)
% [note: riesz kernel are fractional derivatives in dimension 2]
% https://en.wikipedia.org/wiki/Riemann%E2%80%93Liouville_integral
\begin{frame}
FRACTIONAL DERIVATIVES\\
======================

\end{frame}


% recall properties of fourier transform
\begin{frame}
PROPERTIES OF THE FOURIER TRANSFORM\\
===================================

\end{frame}


% definition in the spectral domain (by means of symmetries)
\begin{frame}
SPECTRUM OF THE RIESZ POTENTIAL\\
===============================

\end{frame}


% commutation with zoom
\begin{frame}
COMMUTATION WITH ZOOM\\
=====================

\end{frame}


% continuous interpolation between laplacian, retinex, image, blurry image
\begin{frame}
INTERESTING POINTS IN THE RIESZ SEMiGROUP\\
=========================================

\end{frame}


% first derivatives
\begin{frame}
GET FIRST DERIVATIVES FROM THE LAPLACIAN\\
========================================

\end{frame}


% closely related construction: riesz potentials, riesz transform
% https://en.wikipedia.org/wiki/Riesz_potential
% https://en.wikipedia.org/wiki/Riesz_transform
\begin{frame}
SIMILAR CONSTRUCTIoNS\\
=====================

\end{frame}

\begin{frame}
FORMAL DEFINITION, CONSTANTS\\
============================

\end{frame}


% the Land kernel (particular case for σ=1)
\begin{frame}
THE LAND KERNEL\\
===============

\end{frame}


% implementation: imscript (with fft), imscript (with blur), python
\begin{frame}
IMPLEMENTATION\\
==============

\end{frame}


% comparison table gauss/riesz/land
\begin{frame}
COMPARISON TABLE\\
================

\end{frame}




% 3. applications
\begin{frame}
APPLICATIONS\\
============

\end{frame}


% 3.1. multi-scale retinex (three-gaussians shit, are an approximation of land)
\begin{frame}
MULTI-SCALE RETINEX\\
===================

\end{frame}


% 3.2. - antilaplacian = land * land
\begin{frame}
ANTILAPLACIAN AND LAND\\
======================

\end{frame}


% 3.3. shepard interpolation
% https://en.wikipedia.org/wiki/Inverse_distance_weighting
\begin{frame}
shepard interpolation (idw)\\
===========================

\end{frame}


% 3.4. sobolev fractals, cloud simulation, other textures
\begin{frame}
SOBOLEV FRACTALS, CLOUD SIMULATION\\
==================================

\end{frame}


% gaussian mandelbrot quote
\begin{frame}
CLT AND ITS HEURISTIC CONVERSE\\
==============================

\end{frame}


% conclusion: links to slides, imscript, python function
\begin{frame}
CONCLUSIOn\\
==========

\end{frame}


% colophon: makefile that runs experiments and builds everything
\begin{frame}
COLOPHON\\
========

\end{frame}


\end{document}


% vim:sw=2 ts=2 spell spelllang=en:
